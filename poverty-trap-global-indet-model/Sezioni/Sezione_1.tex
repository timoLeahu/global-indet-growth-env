\section{Introduction}\label{sec:introduction} 
This report is based mainly on a couple of papers, which are correlated: the first one is \cite{caravaggio_nonlinear_2018}, in which I have chosen the a continuous time model and in particular the model briefly described in the subsection 4.1 called \textit{"The Antoci-Russu-Galeotti Model"}, and the second one is \cite{antoci_poverty_2011}, the one containing the full description. Nonetheless other third part papers could be involved and cited. 

As Antoci \cite{antoci_poverty_2011} pointed out in the paper, equilibrium selection in dynamic optimization models with externalities may depend on expectations; that is, given the initial values of the state variables (history), the path followed by the economy maybe determined by the choice of the initial values of the jumping variables. This implies that expectations play a key role in equilibrium selection and in fact global indeterminacy may occur: that is, starting from the same initial values of the state variables, different equilibrium paths can approach different $\omega$-limit sets (for example, different stationary states). In this context, local stability analysis may be misleading, in that it refers to a neighborhood of a stationary state, whereas the initial values of jumping variables do not have to belong to such a neighborhood. In words of Matsuyama \cite{matsuyama_increasing_1991} p. 619:
\textit{“Knowing the local dynamics is not enough, because, for example, demonstrating the uniqueness of the perfect foresight path in a neighborhood of a stationary state does not necessarily rule out the existence of other perfect foresight paths in the large.”}

Although some works on indeterminacy focus on global dynamics and stress the relevance of global analysis, the literature on indeterminacy is almost exclusively based on local analysis, due to the fact that dynamic models exhibiting indeterminacy are often highly nonlinear and difficult to be analyzed globally. Few papers study global indeterminacy in environmental dynamics, see references in \cite{antoci_poverty_2011}. \\
In the present work the authors were not dealing with another important problem in economic dynamics, namely the existence of indifference points in an optimal control problem. Starting from these points, several optimal solutions exist, giving rise to the same value of the objective function. Vice versa, in our context, as it happens to be the case in the literature on indeterminacy, the trajectories followed by the economy are Nash equilibria but do not represent optimal solutions, being the dynamics conditioned by externalities. Therefore, when multiple Nash equilibrium trajectories exist, starting from the same initial values of the state variables, economic agents may select one which is Pareto-dominated by others, due to coordination problems.

The objective of the paper \cite{antoci_poverty_2011} is to highlight the relevance of global indeterminacy in a context in which economic activity depends on the exploitation of a free-access natural resource. They analyzed a growth model with environmental externalities, giving rise to a three-dimensional nonlinear dynamic system (the framework is that introduced by Wirl in \cite{wirl_stability_1997}). In particular they studied the equilibrium growth dynamics of an economy constituted by a continuum of identical agents. At each instant of time $t$, the representative agent produces the output $Y(t)$ by labor $L(t)$, by the accumulated physical capital $K(t)$ and by the stock $E(t)$ of an open-access renewable natural resource. The economy-wide aggregate production $Y(t)$ negatively affects the stock of the natural resource; however, the value of $Y(t)$ is considered as exogenously determined by the representative agent, so that economic dynamics is affected by negative environmental externalities.\footnote{Environmental externalities can affect economic activities especially in developing countries, where property rights tend to be ill-defined and ill-protected, environmental institutions and regulations are weak and natural resources are more fragile than in developed countries, which are located in temperate areas instead than in tropical and sub-tropical regions.}

They assumed that the representative agent’s instantaneous utility, depending on leisure $1-L(t)$ and consumption $C(t)$ of the output $Y(t)$, is represented by the additively non-separable function $\frac{[C(1-L)^{\epsilon}]^{1-\eta} -1}{1-\eta}$. Moreover, we assume that the production technology is represented by the Cobb–Douglas function $[K(t)]^\alpha [L(t)]^\beta [E(t)]^\gamma$, with $\alpha + \beta < 1$ and $\alpha$, $\beta$, $\gamma \,>0$. 

In this context, we show that, if $\alpha + \beta < 1$, the dynamics can admit a locally attracting stationary state (also equilibria or fixed point) $P_1^* = (K_1^*, E_1^*, L^*)$, in fact a \textit{poverty trap}, coexisting with another stationary state $P_2^* = (K_2^*,E_2^*,L^*)$, where $K_1^* < K_2^*$ and $E_1^* < E_2^*$, possessing saddle-point stability.\\ 
Global analysis shows that, under some conditions on the parameters, if the economy starts
from initial values $K_0$ and $E_0$ sufficiently close to $K_1^*$ and $E_1^*$, then there exists a continuum of initial values $L_0^1$ such that the trajectory from $(K_0,E_0,L_0^1)$ approaches $P_1^*$ and a locally unique initial value $L_0^2$ such that the trajectory from $(K_0,E_0,L_0^2)$ approaches $P_2^*$ (see Fig. \ref{fig:local_indeter}). Therefore, their model exhibits local indeterminacy (i.e. there exists a continuum of trajectories leading to $P_1^*$), but also global indeterminacy, since either $P_1^*$ or $P_2^*$ may be selected according to agents’ expectations. Along the trajectories belonging to the basin of attraction of $P_1^*$, over-exploitation
of the natural resource drives the economy towards a \textit{tragedy of commons} scenario.

